\chapter{Conclusion}

Grâce à nos deux modes d'affichage et nos multiples annotations, 
notre outil de visionnage permet de mieux saisir l'état du jeu tel
qu'il est perçu par un robot.
\bigskip

On peut imaginer une interface dédiée aux développeurs qui serait
centrée sur un robot ou une équipe. Cela leur permettrait d'avoir 
des informations précises plus détaillées et complètes 
qu'actuellement.
\bigskip

Notre interface ne traite qu'une seule vidéo à la fois. Pouvoir 
changer de point de vue lors d'un match serait un atout pour 
pouvoir mieux suivre les actions.
\bigskip

Nous avons préféré rester sur une interface plutôt générale mais 
bien fonctionnelle (scroll dans les équipes, pop-up pour le choix 
des annotations et slider). Ajouter les annotations à l'interface 
au fur et à mesure a été moins compliqué que d'ajouter ces outils 
ergonomiques.
\bigskip

Pour ce qui est des annotations, nous avons longtemps parlé de 
faire des ellipses à la place des cercles. 
\bigskip

L'idée était exprimer la dispersion des probabilités de positons 
qu'un robot peut envoyer. Ces ellipses pouvaient aussi représenter
la distorsion de la caméra, nous permettant de mieux comprendre 
l'angle de vision.
\bigskip

Finalement, nous n'avons jamais implémenté l'ellipse, n'ayant de 
la part du robot qu'une simple position (x,y). Nous n'avons pas 
jugé l'implémentation de l'ellipse prioritaire, nous avons préféré
ajouter des annotations.
\bigskip
\newpage

Nous avons pensé à ajouter les informations sur le jeu dans
l'interface (temps restant en touche pour robot exclu, nombre de 
cartons jaune et rouge par robot). Ces informations étaient d'une 
utilité limitée dans les premières vidéos tournées (pas de buts). 
Nous avons trouvé plus pertinent d'afficher à la place les choix 
des annotations.
\bigskip

En revanche, si l'on stocke les \textbf{GCRobotMsg}, il paraît 
plutôt simple d'afficher ces informations. Il suffit de lire le 
message dans la classe \textbf{RobotInformation}, puis lire le 
message grâce aux fonctions proto déjà implémentées.
\bigskip

Ajouter des annotations comme les penalty, la position des 
adversaires ou encore la cible de la frappe est assez rapide.
Ce sont des simples cercles à afficher comme la balle ou la 
position du robot.
\bigskip

Enfin, il aurait été intéressant de pouvoir enregistrer une vidéo 
annotée. Nous avions testé l'enregistrement d'une vidéo à partir 
d'images lors du premier rendu du cahier des besoins. 

Nous n'avons pas adapté notre code de test aux fichiers du client,
mais nous avions réussi à enregistrer à partir d'une liste
d'image.
\bigskip

Au terme de ce travail, nous avons pu créer un outil de
visualisation de match de football robotique qui répond à une
majorité des besoins du client. Notre projet est destiné à évoluer
grâce aux nombreuses extensions possibles.
