
\chapter{Introduction}

La robotique est un domaine en plein essor. Depuis que Deep Blue a battu Kasparov en 1997, le monde s'est rendu compte du potentiel de l'intelligence artificielle.
\bigskip

Au delà d'une seule intelligence, le sport nécessite des capacités physiques. En effet, lors d'un match de football, l'environnement est dynamique, il évolue sans cesse. Un robot qui joue au football se voit donc davantage affronter la complexité du monde réel.
\bigskip

La Robocup est une compétition où depuis 1997 s'affrontent des équipes de robots au football. 
Comme l'intelligence a surpassé l'homme aux échecs et au jeu de Go, le but dans cette compétition est de créer une équipe robotique capable de surpasser une équipe humaine.
Il existe d'autres matchs et tournois en dehors de cette compétition, mais une victoire à la Robocup reste un enjeu majeur pour les équipes.
\bigskip

Notre client est l'équipe de Rhoban évoluant dans la compétition avec des robots humanoïdes kid-size, 

Notre but ici est de créer un outil d'annotation afin de mieux comprendre le match et les comportements des robots. A partir des messages envoyés par les robots et d'un flux vidéo, nous allons donc annoter les images en fonction de ce que le robot perçoit et essaye de faire.
\bigskip

Nous sommes capables d'annoter la position et la direction du robot. Également, on ajoute pour un robot à la fois, sa perception de la balle et l'historique de ses positions. Enfin, nous sommes capable d'afficher la position que souhaite atteindre un robot (l'intention).
\bigskip

Nous avons créer deux façons d'afficher la vidéo. Tout d'abord un outil simple permettant de visualiser une vidéo annotée selon des paramètres prédéfinis par l'utilisateur.
\bigskip

Ensuite, nous pouvons regarder la vidéo grâce à un interface graphique. On peut se déplacer en avant et en arrière grâce à un slider et changer les annotations en cours de visionnage.


