\chapter{Besoins}

Ce projet consiste à proposer une solution logicielle pour 
incruster les données d'un match de football robotique sur un 
flux vidéo.

Ce logiciel devra répondre aux besoins qui ont été définis au 
préalable avec les clients, des besoins \textit{fonctionnels} et 
\textit{non fonctionnels}.
\bigskip

La liste des besoins de notre cahier des besoins n'était pas 
bonne car notre compréhension du code fourni par le client était 
à l'époque très faible. Nous rencontrions des problèmes avec 
\textbf{Catkin} (Build system, fourni par le framework ROS 
spécifique pour l'écriture de logiciel de robot), ce qui nous 
empêchait de compiler. Ces problèmes ont rapidement été 
surmontés, ce qui nous a permis de mieux cerner le projet. 
 
\section{Fonctionnels}

Comme tous les besoins n'ont pas une importance similaire, nous 
avons décidé de les séparer en 3 niveaux de priorité.

\subsection{Priorité 1}

Les besoins fonctionnels de priorité 1 sont indispensables au 
fonctionnement du programme. 
\bigskip

\begin{description}
    \item[Interface]: Les clients désiraient une interface dans 
    laquelle serait affichée une vidéo annotée. Depuis cette 
    interface, il doit être possible de modifier les choix 
    d'annotations.\\
    
    Nous avons donc créé une interface en utilisant le framework 
    \textbf{Qt}. Elle doit communiquer avec le package qui 
    s'occupe d'incruster les annotations sur la vidéo. Elle peut 
    ainsi récupérer les images annotées et les afficher.
    \bigskip
    
    L'interface affiche aussi des informations utiles (le temps 
    et le score) pour le match en cours. Elle propose des moyens 
    d'interaction pour personnaliser les annotations (menu 
    Annotation Choice). On peut contrôler le visionnage de la 
    vidéo grâce aux boutons Pause, Fast Forward et au Slider.
    \bigskip
    \item[Annotation de la position et de la direction] : 
    Certaines annotations sont indispensables pour comprendre les
    actions des robots. Il est important d'incruster sur la vidéo
    l'endroit où chaque robot pense être, ainsi que sa direction.
    \bigskip
    
    Nous avons choisi de représenter la position par un point de 
    taille fixe de la couleur de l'équipe du robot (une couleur 
    par équipe).  La direction est une flèche de même couleur que
    le point qui part du centre du point position et dans la
    direction indiquée par les données du robot.
    \bigskip
    
    Nous avons implémenté ces solutions en utilisant les fonction
    de dessin d'\textbf{OpenCV}, qui ne demandent que très peu de
    puissance de calcul \ref{vitesse_annotation}. 
    \bigskip
    
    \item[Compréhension et appropriation du code du client (et 
    Catkin)]: Ce fut un des gros problèmes du début de projet. 
    Nous avons eu beaucoup de mal à compiler le code client 
    fourni, qui était nécessaire pour l'avancée du projet 
    (notamment pour recevoir les messages des robots et les lire;
    il sert aussi à décomposer la vidéo en images et permet de
    transformer la position du robot en point sur l'image en 
    prenant en compte les paramètres de la caméra). Le problème 
    venait des dépendances avec Catkin, que nous ne savions pas 
    comment utiliser.
    \bigskip
    
    Finalement, après de nombreux essais pour faire fonctionner 
    CMake, Catkin et les autres dépendances, et avec de l'aide de
    nos clients, nous avons réussi à compiler le code fourni.
    Nous avons ainsi pu intégrer notre code dans ce système de 
    compilation. Plus tard, nous avons passé les modules clients 
    ainsi que le paquet Catkin en  submodule de notre dépot git 
    pour pouvoir les mettre à jour lorsque c'est nécessaire.
\end{description}

Ces besoins sont la base du projet, et étaient le minimum à 
atteindre pour avoir un programme utilisable. 
\bigskip

Nous avions pensé devoir calculer l'équation de plan permettant  
de récupérer un point sur l'image à partir d'une position sur le 
terrain. Ces deux articles \cite{ref1} et \cite{ref2} nous ont 
offerts une première approche sur le problème.

Lorsque nous avons pu lire le code fourni par le client, nous 
avons vu comment utiliser ses fonctions qui prenaient déjà en 
compte l'équation et nous n'avons donc pas eu à travailler 
dessus. 

\subsection{Priorité 2}

Ces besoins dépendent de ceux de priorité 1, mais sont 
indispensables au bon fonctionnement du programme.
\bigskip

\begin{description}
    \item[Relier l'annotation des images à l'interface]: 
    L'interface offre du confort à l'utilisateur, notamment pour 
    choisir les annotations. L'annotation de l'image se fait dans
    le module \textbf{annotateImage}. Pour pouvoir afficher les
    images annotées dans l'interface, il a fallu mettre en place 
    une communication entre ces deux modules.
    \bigskip
    
     Dans la première moitié du projet, nous avons eu beaucoup de
     difficultés à relier ces deux parties. Ceci était
     principalement dû à nos (nombreux) problèmes d'architecture 
     précédant le rendu de la V1. Notre nouvelle architecture, 
     plus robuste, nous a permis de résoudre ce problème de 
     communication inter-modules. 
    \bigskip
    
    Auparavant, nous appelions un manager depuis un autre manager
    d'annotation. Le premier manager prenait alors la main, et ne
    laissait pas le second s'exécuter. Nous avons résolu ce 
    problème avec un QTimer qui met périodiquement le manager à 
    jour.
    \bigskip
    
    Nous appelons donc l'annotation d'images périodiquement pour 
    rafraîchir l'image \textit{augmentée} dans l'interface. 
    Celle-ci communique avec le manager pour choisir les 
    annotations à incruster.
    \bigskip
    
    \newpage
    \item[Avoir une architecture permettant une adaptabilité du 
    projet]: Les informations envoyées par les robots peuvent 
    changer. À l'avenir, ils enverront peut être de nouvelles 
    informations que nous ne pouvons pas afficher actuellement 
    (comme l'indication du périmètre de sécurité autour d'un 
    robot pour un coup-franc par exemple !). Nous devions faire 
    en sorte qu'il soit simple d'ajouter de nouvelles annotations
    au code. 
    \bigskip
    
   Si on souhaite ajouter un nouveau type d'annotation, il suffit
   d'ajouter une fonction d'incrustation dans
   \textit{annotation.cpp} (paquet \textit{annotateImage}), et
   d'ajouter un champ pour l'annotation dans le fichier json 
   \textit{annotation\_settings.json} (qui gère les caractériques
   des annotations (taille, couleur, ...).
   \bigskip
   
    \item[Permettre à l'utilisateur de personnaliser ses
    annotations et de faire des choix]: L'utilisateur doit être 
    en mesure de configurer les paramètres esthétiques des 
    annotations (taille, couleur ...). En effet, lors d'un match 
    avec plusieurs robots, afficher toutes les annotations pour 
    tous les robots rendrait la vidéo complètement illisible tant
    il y aurait de contenu à l'écran. 
    \bigskip
    
    De plus, ne pas avoir de personnalisation des annotations
    rendrait le logiciel très peu flexible, et restreindrait
    l'éventail des utilisations possibles.
    C'est pourquoi nous avons mis en place un fichier
    \textit{annotation\_settings.json} contenant les paramètres
    principaux de chaque annotation, et qui est apte à être
    modifié pour toute utilisation souhaitée.
    
\end{description}

\subsection{Priorité 3}

Ces besoins ne sont pas indispensables au fonctionnement du
programme, mais lui permettent de gagner en ergonomie.
\bigskip

\begin{description}
    \item[Faciliter le déplacement dans la vidéo]: Comme dans la
    plupart des logiciels de lecture de vidéo, il est utile de
    pouvoir se déplacer dans la vidéo. Nous avons solutionné ce
    problème en implémentant un bouton \textit{play/pause}. Nous
    avons également ajouté la possibilité de passer en mode 
    \textit{fast-forward} pour avancer plus rapidement, ainsi
    qu'un slider sous la vidéo, qui va permettre de se déplacer 
    dans la vidéo et de visualiser son avancement.
    \bigskip
    
    \item[Permettre à l'utilisateur de personnaliser 
    l'incrustation]: Ce besoin est un peu englobé dans le besoin 
    de l'interface, mais n'était pas indispensable pour obtenir 
    une interface fonctionnelle. 
    \bigskip
    
    Il est souhaitable que l'utilisateur puisse faire des choix 
    de manière intuitive depuis l'interface pour personnaliser 
    son expérience. Par exemple, choisir les annotations à 
    afficher ou non. 
    
    Nous avons implémenté de nombreuses solutions à ce besoin 
    dans l'interface, comme le menu Annotation Choice qui permet 
    de choisir quelles annotations on souhaite afficher, ou 
    encore la trace de quel robot on souhaite afficher.
    
    \item[Ajouter de nouvelles annotations]: L'incrustation de la
    position et de la direction est un besoin essentiel. Cela
    représente toutefois peu d'informations, et il serait 
    souhaitable d'en proposer d'avantage. 
    \bigskip
    
    Grâce à notre l'architecture, nous avons pu rajouter de 
    nombreuses annotations comme par exemple une trace des 
    dernières positions d'un robot, ou la position de la balle 
    telle que perçue par un robot. On choisit le robot pour 
    lequel afficher l'annotation car l'affichage serait trop 
    surchargé si on l'affichait pour tous les robots.
    \bigskip
    
    \item[Afficher l'état de la partie sur l'interface]: Les 
    clients souhaitaient voir les informations sur les équipes, 
    les robots, et le match dans l'interface. 
    \bigskip
    
    Nous avons choisi d'afficher le score ainsi que la liste des 
    robots de chaque équipe, mais aussi quelle annotation est 
    activée pour chaque robot.
\end{description}

\newpage

\section{Besoins non fonctionnels}

\begin{description}
    \item[Dépendances à jour]: Pour faire tourner le projet, nous
    utilisons de nombreuses dépendances (qui sont listées dans le
    README d'installation) comme OpenCV 3, QT 5.9.5, mais aussi 
    Catkin, CMake, python ou encore protobuf. 
    \bigskip
    
    Sans ces dépendances, on ne peut pas compiler le projet 
    (suivre le \textit{README} pour configurer).
    \bigskip
    
    \item[Vidéo du match]: La vidéo du match doit être de 
    résolution 640x480, les formats avi, mov et mp4 sont tous 
    acceptés. 
    \bigskip
    
    La caméra doit rester statique pendant tout l'enregistrement 
    (on ne peut pas mettre à jour l'équation de plan), et ne pas 
    avoir de trop grosse déformations (par exemple l'effet 
    fisheye d'une caméra grand angle va générer des erreurs).
    \bigskip
    
    \item[Logs et fichiers de configuration]: La vidéo du match 
    doit se trouver dans le même dossier que les logs du match, 
    ainsi que les fichiers de configuration json. 
    \bigskip
    
    Voir l'exemple avec les dossiers présents dans le google 
    drive donné dans le README, qui contient les fichiers et les 
    vidéos de matchs de football robotique.
    
    \item[Equipes de robots qui respectent la hiérarchie des 
    logs]: Tous les robots ne communiquent pas de la même 
    manière, et tous les logs ne suivent pas la même 
    architecture. 
    
    Nos clients ont mis en place une architecture de logs. Les 
    logs fournis doivent donc suivre l'architecture définie par 
    le client pour être exploitables par notre logiciel.
\end{description}

